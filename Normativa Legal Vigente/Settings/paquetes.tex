% =========================================
% 1. Documento y tema
% =========================================
\usetheme{Singapore}               % Tema

% =========================================
% 2. Codificación y lenguaje
% =========================================
\usepackage[utf8]{inputenc}     % Codificación de entrada UTF-8
\usepackage[T1]{fontenc}        % Codificación de salida de fuentes
\usepackage[spanish]{babel}     % Traduce términos al español (p. ej. “Figura”, “Tabla”)

% =========================================
% 3. Hipervínculos
% =========================================
\usepackage{hyperref}           % Soporte para enlaces intern​os y externos

% =========================================
% 4. Gráficos y dibujos
% =========================================
\usepackage{graphicx}           % Incluir imágenes (\includegraphics)
\usepackage{tikz}               % Dibujos vectoriales
\usetikzlibrary{
    arrows.meta,                 % Flechas avanzadas
    positioning,                 % Posicionamiento relativo
    shapes.geometric             % Formas geométricas
}
\usepackage{pgf-umlcd}          % Diagramas UML (opcional)

% =========================================
% 5. Matemáticas
% =========================================
\usepackage{amsmath,amssymb,amsfonts}  % Entornos y símbolos matemáticos

% =========================================
% 6. Tablas y columnas
% =========================================
\usepackage{array}              % Extiende entorno tabular
\usepackage{tabularx}           % Columnas de ancho variable
\usepackage{colortbl}           % Colorear filas/columnas
\usepackage{booktabs}           % Reglas horizontales mejoradas
\usepackage{multicol}           % Listas y contenido en varias columnas
\usepackage{caption}            % Personalizar pies de foto/tablas

% =========================================
% 7. Presentación y alineación de texto
% =========================================
\usepackage{ragged2e}           % Justificado avanzado (\justifying)

% =========================================
% 8. Colores personalizados
% =========================================
\usepackage{xcolor}
% Colores corporativos
\definecolor{azul}{RGB}{0,79,159}
\definecolor{grisClaro}{RGB}{245,245,245}
\definecolor{azulClaro}{RGB}{222,235,247}
\definecolor{verdeClaro}{RGB}{178,204,93}
\definecolor{verdeOscuro}{RGB}{130,180,100}
\definecolor{celeste}{RGB}{107,191,183}
\definecolor{morado}{RGB}{123,104,180}
% Color para texto estructural
\definecolor{mygreen}{RGB}{0,128,0}

% =========================================
% 9. Personalización de colores en Beamer
% =========================================
\setbeamercolor{normal text}{fg=darkgray}  % Texto principal en gris oscuro
\setbeamercolor{structure}{fg=darkgray}     % Títulos, secciones, etc. en verde
\setbeamercolor{frametitle}{fg=darkgray}      % Título de diapositiva
\setbeamercolor{subframetitle}{fg=white,bg=morado}  % Subtítulos

% =========================================
% 10. Unidades y notación científica
% =========================================
\usepackage{siunitx}            % Formato de números y unidades
\sisetup{
    output-exponent-marker = \text{e},   % Notación científica 1e3
    per-mode = symbol,                   % Unidades con “/”
    separate-uncertainty = true,         % Incertidumbre entre paréntesis
    multi-part-units = single            % Unidades compuestas
}

% =========================================
% 11. Cuadros de color y resaltado
% =========================================
\usepackage{tcolorbox}          % Bloques de color personalizados

% =========================================
% 12. Comandos personalizados
% =========================================
\usepackage{fontawesome}        % Iconos (✔, ⚠, ℹ) en listas
\newcommand{\warningblock}[1]{%
    \begin{alertblock}{¡ATENCIÓN!}#1\end{alertblock}%
}
\newcommand{\tipblock}[1]{%
    \begin{block}{CONSEJO TÉCNICO}#1\end{block}%
}
\newcommand{\noteblock}[1]{%
    \begin{block}{NOTA IMPORTANTE}#1\end{block}%
}
\newcommand{\checkitem}{\item[\faCheckCircle]}
\newcommand{\alertitem}{\item[\faExclamationTriangle]}
\newcommand{\infoitem}{\item[\faInfoCircle]}

% =========================================
% 13. Definir tema de pie de página
% =========================================
\setbeamertemplate{footline}{%
    \leavevmode
    \hbox{%
        \begin{beamercolorbox}[wd=\paperwidth,sep=1ex,left]{footline}%
            \hfill{\footnotesize | © CAPACITACIÓN S.ZÚÑIGA. SANTIAGO DE CHILE}%
        \end{beamercolorbox}%
    }%
}

% =========================================
% Ahora el preámbulo es coherente y organizado.
% =========================================

% ======================================================
% 1. Clase de documento y opciones globales
% ======================================================
\documentclass[spanish]{beamer}
% - “spanish”: localiza automáticamente textos de Beamer en español.

% ======================================================
% 2. Preámbulo común
% ======================================================
% =========================================
% 1. Documento y tema
% =========================================
\usetheme{Singapore}               % Tema

% =========================================
% 2. Codificación y lenguaje
% =========================================
\usepackage[utf8]{inputenc}     % Codificación de entrada UTF-8
\usepackage[T1]{fontenc}        % Codificación de salida de fuentes
\usepackage[spanish]{babel}     % Traduce términos al español (p. ej. “Figura”, “Tabla”)

% =========================================
% 3. Hipervínculos
% =========================================
\usepackage{hyperref}           % Soporte para enlaces intern​os y externos

% =========================================
% 4. Gráficos y dibujos
% =========================================
\usepackage{graphicx}           % Incluir imágenes (\includegraphics)
\usepackage{tikz}               % Dibujos vectoriales
\usetikzlibrary{
    arrows.meta,                 % Flechas avanzadas
    positioning,                 % Posicionamiento relativo
    shapes.geometric             % Formas geométricas
}
\usepackage{pgf-umlcd}          % Diagramas UML (opcional)

% =========================================
% 5. Matemáticas
% =========================================
\usepackage{amsmath,amssymb,amsfonts}  % Entornos y símbolos matemáticos

% =========================================
% 6. Tablas y columnas
% =========================================
\usepackage{array}              % Extiende entorno tabular
\usepackage{tabularx}           % Columnas de ancho variable
\usepackage{colortbl}           % Colorear filas/columnas
\usepackage{booktabs}           % Reglas horizontales mejoradas
\usepackage{multicol}           % Listas y contenido en varias columnas
\usepackage{caption}            % Personalizar pies de foto/tablas

% =========================================
% 7. Presentación y alineación de texto
% =========================================
\usepackage{ragged2e}           % Justificado avanzado (\justifying)

% =========================================
% 8. Colores personalizados
% =========================================
\usepackage{xcolor}
% Colores corporativos
\definecolor{azul}{RGB}{0,79,159}
\definecolor{grisClaro}{RGB}{245,245,245}
\definecolor{azulClaro}{RGB}{222,235,247}
\definecolor{verdeClaro}{RGB}{178,204,93}
\definecolor{verdeOscuro}{RGB}{130,180,100}
\definecolor{celeste}{RGB}{107,191,183}
\definecolor{morado}{RGB}{123,104,180}
% Color para texto estructural
\definecolor{mygreen}{RGB}{0,128,0}

% =========================================
% 9. Personalización de colores en Beamer
% =========================================
\setbeamercolor{normal text}{fg=darkgray}  % Texto principal en gris oscuro
\setbeamercolor{structure}{fg=darkgray}     % Títulos, secciones, etc. en verde
\setbeamercolor{frametitle}{fg=darkgray}      % Título de diapositiva
\setbeamercolor{subframetitle}{fg=white,bg=morado}  % Subtítulos

% =========================================
% 10. Unidades y notación científica
% =========================================
\usepackage{siunitx}            % Formato de números y unidades
\sisetup{
    output-exponent-marker = \text{e},   % Notación científica 1e3
    per-mode = symbol,                   % Unidades con “/”
    separate-uncertainty = true,         % Incertidumbre entre paréntesis
    multi-part-units = single            % Unidades compuestas
}

% =========================================
% 11. Cuadros de color y resaltado
% =========================================
\usepackage{tcolorbox}          % Bloques de color personalizados

% =========================================
% 12. Comandos personalizados
% =========================================
\usepackage{fontawesome}        % Iconos (✔, ⚠, ℹ) en listas
\newcommand{\warningblock}[1]{%
    \begin{alertblock}{¡ATENCIÓN!}#1\end{alertblock}%
}
\newcommand{\tipblock}[1]{%
    \begin{block}{CONSEJO TÉCNICO}#1\end{block}%
}
\newcommand{\noteblock}[1]{%
    \begin{block}{NOTA IMPORTANTE}#1\end{block}%
}
\newcommand{\checkitem}{\item[\faCheckCircle]}
\newcommand{\alertitem}{\item[\faExclamationTriangle]}
\newcommand{\infoitem}{\item[\faInfoCircle]}

% =========================================
% 13. Definir tema de pie de página
% =========================================
\setbeamertemplate{footline}{%
    \leavevmode
    \hbox{%
        \begin{beamercolorbox}[wd=\paperwidth,sep=1ex,left]{footline}%
            \hfill{\footnotesize | © CAPACITACIÓN S.ZÚÑIGA. SANTIAGO DE CHILE}%
        \end{beamercolorbox}%
    }%
}

% =========================================
% Ahora el preámbulo es coherente y organizado.
% =========================================

% Carga un archivo externo con TODOS los \usepackage{},
% definiciones de colores, comandos, etc. centralizados.

% ======================================================
% 3. Inicio del documento
% ======================================================
\begin{document}

    % ------------------------------------------------------
    % 3.1 Configuración inicial (título, autor, fecha…)
    % ------------------------------------------------------
    \newcommand{\fecha}{\DTMDate{2024-12-09}}
\newcommand{\titulo}{}
\newcommand{\autor}{
	Sebastián Zúñiga
}
%Renuevo comandos
\renewcommand{\contentsname}{Contenido}
\renewcommand{\tablename}{\bfseries Tabla}
\renewcommand{\figurename}{\bfseries Figura}
\renewcommand{\thefootnote}{\textcolor{grayblack}{\arabic{footnote}}}

    % En este archivo define \title{…}, \author{…}, \date{…},
    % y ajustes de tema (colores, footline, headline, etc.)

    % ======================================================
    % 4. Portada “canónica”
    % ======================================================
    % =========================================
% 1. Información de la presentación
% =========================================
%   Coloca estas líneas EN EL PREÁMBULO, justo después de cargar
%   los paquetes y antes de \begin{document}

% Título de la presentación:
% - Opción corta para el encabezado (aparece en la barra de navegación)
% - Opción larga para la diapositiva de título
\title[NCh 440/1 – 2014]{Normativa Legal Vigente en el Transporte Vertical}

% Autor(es):
% - Opción corta para el pie de página o navegación
% - Nombre completo en la diapositiva de título
\author[S.\ Zúñiga]{Sebastián Alfredo Zúñiga Alfaro}

% Institución o entidad:
% - Texto que aparece debajo del autor en la portada
\institute[]{Capacitación}

% Fecha:
% - Opción corta (por ejemplo, día/mes/año) para la barra de navegación
% - Texto completo en la portada
\date[23/4/25]{23 de abril de 2025}

    % Diapositiva de portada estándar generada con \maketitle o \titlepage

    % ======================================================
    % 5. Portada “a medida” sin cabecera ni pie
    % ======================================================
    {
    % Dentro de un grupo local para no alterar el resto
        \setbeamertemplate{headline}{}   % Suprime la cabecera
        \setbeamertemplate{footline}{}   % Suprime el pie de página
        \begin{frame}[plain]             % “plain” = sin decoraciones
            \vspace*{-1cm}                 % Eleva el contenido
            \begin{columns}
                \column{0.45\textwidth}       % Espacio reservado para logo
                \vspace*{-6.5cm}              % Ajuste vertical del logo
                %\includegraphics[width=4cm]{Imágenes/FM}
                % Uncomment y ajusta la ruta del logo
            \end{columns}
            \titlepage                     % Genera automáticamente título, autor, fecha
        \end{frame}
    }

    % ======================================================
    % 6. Índice automático al comenzar cada sección
    % ======================================================
    \AtBeginSection[]{
        \begin{frame}{Contenido}
            \begin{multicols}{2}          % Divide el índice en 2 columnas
                \tableofcontents[currentsection]
            \end{multicols}
        \end{frame}
    }

    % ======================================================
    % 7. Contenido principal: secciones
    % ======================================================
    \begin{frame}[t]{\textbf{Uso del presente material}}
    % ---------------------------------------
    % Opciones y título de la diapositiva
    % ---------------------------------------
    % [t] : alinea el contenido arriba
    % {Uso del presente material} : título de la diapositiva

    % ---------------------------------------
    % Columnas para maquetación
    % ---------------------------------------
    \begin{columns}[onlytextwidth]  % ajusta las columnas al ancho de texto
        \column{1\textwidth}         % esta columna ocupa el 100 % del ancho
        \textbf{Para objetivos de capacitación únicamente:}\\
        \vspace{0.5cm}               % espacio vertical entre título y párrafo
        \centering
        % -------------------------------------
        % Texto justificado (requiere ragged2e)
        % -------------------------------------
        \justifying
        La información en este paquete es sólo para fines de capacitación.
        Cualquier información técnica relativa a un producto es relevante sólo
        a la capacitación, y no debe ser utilizado por el destinatario o
        transmitido a otra persona para su uso en el diseño, fabricación,
        pruebas, instalación, inspección, mantenimiento, modernización o
        reparación del producto al que se refiere.
    \end{columns}
\end{frame}



    \section{Aspectos de seguridad}
\begin{frame}{Aspectos de seguridad}
	\begin{itemize}
		\item ¿Cuál es la altura máxima, de los botones de comando, en la botonera de cabina?
		\item ¿Qué distancia debe existir entre elementos móviles a un ascensor adyacente?
		\item ¿Cuántos lux deben existir en el recorrido?
		\item ¿A qué distancia debe quedar el botón de la parada de emergencia, en un equipo sin sala de máquinas?
		\item ¿Desde qué dimensión de cabina se debe instalar un espejo?
	\end{itemize}
\end{frame}

%------------------------------------------------
\section{Preliminar}
\begin{frame}{Preliminar}
	% Slide original sin contenido adicional
\end{frame}

%------------------------------------------------
\section{Contenido \& Objetivo}
\begin{frame}{Contenido \& Objetivo}
	% En la diapositiva original aparecían los títulos "Contenido" y "Objetivo"
	% Sin lista explícita, se conservan como título único.
\end{frame}

%------------------------------------------------
\section{Bibliografía}
\begin{frame}{Bibliografía}
	% Referencias originales en la diapositiva
\end{frame}

%------------------------------------------------
\section{Registro nacional de profesionales}
\begin{frame}{Registro nacional de instaladores, mantenedores y certificadores}
	\begin{itemize}
		\item Instaladores I, II y III categoría
		\item Mantenedores única categoría
		\item Certificadores I y II categoría
	\end{itemize}
\end{frame}

%------------------------------------------------
\section{Normas técnicas}
\begin{frame}{Normas técnicas}
	\begin{itemize}
		\item Norma técnica NTM/NCh
		\item Norma técnica MINVU
		\item Norma Chilena INN
		\item Elevador debe cumplir norma técnica vigente para certificarse
	\end{itemize}
\end{frame}

%------------------------------------------------
\section{LGUC}
\begin{frame}{Requisitos de la Ley General de Urbanismo y Construcción}
	L.G.U.C. 4.1.7 numeral 3 y 4.1.11\,f
\end{frame}

%------------------------------------------------
\section{NCh 440/1:2014}
\subsection{Punto 5: Caja de elevadores}
\begin{frame}{Caja de elevadores NCh 440/1:2014. Punto 5}
	% Primer pase de diapositiva
\end{frame}

\begin{frame}{Caja de elevadores NCh 440/1:2014. Punto 5}
	% Diapositiva duplicada en original
\end{frame}

\subsection{Ejercicio intermedio}
\begin{frame}{Analice cuanto avanza su conocimiento!}
	Comente los aspectos normativos asociados a las imágenes.
	
	% Marcadores para imágenes originales
	\begin{center}
		% Aquí se insertan las imágenes de apoyo
		% \includegraphics[width=0.45\textwidth]{imagen1.ext}
		% \includegraphics[width=0.45\textwidth]{imagen2.ext}
	\end{center}
\end{frame}

\subsection{Punto 6: Sala de máquinas y cuadro de control}
\begin{frame}{Sala de máquinas y cuadro de control NCh 440/1:2014. Punto 6}
\end{frame}

\begin{frame}{Sala de máquinas y cuadro de control NCh 440/1:2014. Punto 6}
	% Diapositiva duplicada en original
\end{frame}

\subsection{Punto 7: Puertas de acceso en pisos}
\begin{frame}{Puertas de acceso en pisos NCh 440/1:2014. Punto 7}
\end{frame}

\subsection{Punto 8: Cabina y contrapeso}
\begin{frame}{Cabina y contrapeso NCh 440/1:2014. Punto 8}
\end{frame}

\begin{frame}{Analice cuanto avanza su conocimiento!}
	¿Cuál es el tamaño de la ventilación del cuarto de máquinas?
	
	\bigskip
	\textbf{0,1\,m² \quad (20×50\,cm²)}
\end{frame}

\begin{frame}{Cabina y contrapeso NCh 440/1:2014. Punto 8}
	% Diapositiva duplicada en original
\end{frame}

\subsection{Punto 9: Suspensión y seguridad}
\begin{frame}{Suspensión, compensación, paracaídas, limitador de velocidad.\\NCh 440/1:2014. Punto 9}
\end{frame}

\begin{frame}{Suspensión, compensación, paracaídas, limitador de velocidad.\\NCh 440/1:2014. Punto 9}
	% Diapositiva duplicada en original
\end{frame}

\subsection{Punto 10: Guías y amortiguadores}
\begin{frame}{Guías, amortiguadores y dispositivo de final de recorrido.\\NCh 440/1:2014. Punto 10}
\end{frame}

\subsection{Punto 11: Holguras}
\begin{frame}{Holguras entre cabina y paredes de la caja de elevadores,\\y entre la cabina y el contrapeso. NCh 440/1:2014. Punto 11}
\end{frame}

\subsection{Punto 12: Máquina}
\begin{frame}{Máquina NCh 440/1:2014. Punto 12}
\end{frame}

\subsection{Punto 13: Instalación y aspectos eléctricos}
\begin{frame}{Instalación y aspectos eléctricos. NCh 440/1:2014. Punto 13}
\end{frame}

\begin{frame}{Instalación y aspectos eléctricos. NCh 440/1:2014. Punto 13}
	% Diapositiva duplicada en original
\end{frame}

\subsection{Punto 14: Protección eléctrica}
\begin{frame}{Protección contra fallas eléctricas, controles, prioridades.\\NCh 440/1:2014. Punto 14}
\end{frame}

\subsection{Punto 15: Rótulo de instrucciones}
\begin{frame}{Rótulo de instrucciones de operaciones.\\NCh 440/1:2014. Punto 15}
\end{frame}

\begin{frame}{Rótulo de instrucciones de operaciones.\\NCh 440/1:2014. Punto 15}
	% Diapositiva duplicada en original
\end{frame}

\subsection{Punto 16: Inspecciones y mantenimiento}
\begin{frame}{Inspecciones, ensayos, registros, mantenimiento.\\NCh 440/1:2014. Punto 16}
\end{frame}

%------------------------------------------------
\section{Otras normas}
\begin{frame}{Requisitos mínimos de diseño, instalación y operación para ascensores\\electromecánicos frente a sismos. NCh3362:2014}
\end{frame}

\begin{frame}{OGUC (D 47)}
	“…ser comprensibles, utilizables y practicables por todas las personas,\\
	en condiciones de seguridad y comodidad, de la forma más autónoma\\
	y natural posible.”
\end{frame}

\begin{frame}{Inclusión Ley 20.422}
	Si las edificaciones y obras señaladas en este inciso contaren con ascensores, estos deberán tener capacidad suficiente para transportar a las personas con discapacidad de conformidad a la normativa vigente.
\end{frame}

\begin{frame}{Regulación de ascensores regenerativos}
	Los ascensores electromecánicos verticales que requieran instalarse en los edificios señalados en los números 1) y 2) de la letra a) de este numeral, corresponderán a ascensores del tipo regenerativo, entendiéndose por éstos a aquellos que durante el movimiento ascendente o descendente de la cabina o el contrapeso generan energía eléctrica. Igualmente, se conformarán como ascensores del tipo regenerativo aquellos ascensores electromecánicos instalados en estos mismos edificios, que en su alteración o transformación, consideren el cambio de la velocidad nominal, la carga nominal y/o la masa de la cabina; o consideren el cambio o sustitución de la máquina o el sistema de control.
\end{frame}

\begin{frame}{Edificios afectados}
	Edificios de vivienda de más de 6 niveles, sin tener recinto de uso exclusivo.\\
	Destinados a otros usos de 5 o más niveles.
\end{frame}

\begin{frame}{Regenerativo}
	% Título único, sin contenido adicional
\end{frame}
    

    % (Opcional) Slide final con portada para cierre
    \begin{frame}
        \titlepage
    \end{frame}

\end{document}

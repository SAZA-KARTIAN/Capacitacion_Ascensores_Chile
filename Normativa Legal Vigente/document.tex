% ======================================================
% 1. Clase de documento y opciones globales
% ======================================================
\documentclass[spanish]{beamer}
% - “spanish”: localiza automáticamente textos de Beamer en español.

% ======================================================
% 2. Preámbulo común
% ======================================================
% =========================================
% 1. Documento y tema
% =========================================
\usetheme{Madrid}               % Tema

% =========================================
% 2. Codificación y lenguaje
% =========================================
\usepackage[utf8]{inputenc}     % Codificación de entrada UTF-8
\usepackage[T1]{fontenc}        % Codificación de salida de fuentes
\usepackage[spanish]{babel}     % Traduce términos al español (p. ej. “Figura”, “Tabla”)

% =========================================
% 3. Hipervínculos
% =========================================
\usepackage{hyperref}           % Soporte para enlaces intern​os y externos

% =========================================
% 4. Gráficos y dibujos
% =========================================
\usepackage{graphicx}           % Incluir imágenes (\includegraphics)
\usepackage{tikz}               % Dibujos vectoriales
\usetikzlibrary{
    arrows.meta,                 % Flechas avanzadas
    positioning,                 % Posicionamiento relativo
    shapes.geometric             % Formas geométricas
}
\usepackage{pgf-umlcd}          % Diagramas UML (opcional)

% =========================================
% 5. Matemáticas
% =========================================
\usepackage{amsmath,amssymb,amsfonts}  % Entornos y símbolos matemáticos

% =========================================
% 6. Tablas y columnas
% =========================================
\usepackage{array}              % Extiende entorno tabular
\usepackage{tabularx}           % Columnas de ancho variable
\usepackage{colortbl}           % Colorear filas/columnas
\usepackage{booktabs}           % Reglas horizontales mejoradas
\usepackage{multicol}           % Listas y contenido en varias columnas
\usepackage{caption}            % Personalizar pies de foto/tablas

% =========================================
% 7. Presentación y alineación de texto
% =========================================
\usepackage{ragged2e}           % Justificado avanzado (\justifying)

% =========================================
% 8. Colores personalizados
% =========================================
\usepackage{xcolor}
% Colores corporativos
\definecolor{azul}{RGB}{0,79,159}
\definecolor{grisClaro}{RGB}{245,245,245}
\definecolor{azulClaro}{RGB}{222,235,247}
\definecolor{verdeClaro}{RGB}{178,204,93}
\definecolor{verdeOscuro}{RGB}{130,180,100}
\definecolor{celeste}{RGB}{107,191,183}
\definecolor{morado}{RGB}{123,104,180}
% Color para texto estructural
\definecolor{mygreen}{RGB}{0,128,0}

% =========================================
% 9. Personalización de colores en Beamer
% =========================================
\setbeamercolor{normal text}{fg=darkgray}  % Texto principal en gris oscuro
\setbeamercolor{structure}{fg=darkgray}     % Títulos, secciones, etc. en verde
\setbeamercolor{frametitle}{fg=darkgray}      % Título de diapositiva
\setbeamercolor{subframetitle}{fg=white,bg=morado}  % Subtítulos

% =========================================
% 10. Unidades y notación científica
% =========================================
\usepackage{siunitx}            % Formato de números y unidades
\sisetup{
    output-exponent-marker = \text{e},   % Notación científica 1e3
    per-mode = symbol,                   % Unidades con “/”
    separate-uncertainty = true,         % Incertidumbre entre paréntesis
    multi-part-units = single            % Unidades compuestas
}

% =========================================
% 11. Cuadros de color y resaltado
% =========================================
\usepackage{tcolorbox}          % Bloques de color personalizados

% =========================================
% 12. Comandos personalizados
% =========================================
\usepackage{fontawesome}        % Iconos (✔, ⚠, ℹ) en listas
\newcommand{\warningblock}[1]{%
    \begin{alertblock}{¡ATENCIÓN!}#1\end{alertblock}%
}
\newcommand{\tipblock}[1]{%
    \begin{block}{CONSEJO TÉCNICO}#1\end{block}%
}
\newcommand{\noteblock}[1]{%
    \begin{block}{NOTA IMPORTANTE}#1\end{block}%
}
\newcommand{\checkitem}{\item[\faCheckCircle]}
\newcommand{\alertitem}{\item[\faExclamationTriangle]}
\newcommand{\infoitem}{\item[\faInfoCircle]}

% =========================================
% 13. Definir tema de pie de página
% =========================================
\setbeamertemplate{footline}{%
    \leavevmode
    \hbox{%
        \begin{beamercolorbox}[wd=\paperwidth,sep=1ex,left]{footline}%
            \hfill{\footnotesize | © CAPACITACIÓN S.ZÚÑIGA. SANTIAGO DE CHILE}%
        \end{beamercolorbox}%
    }%
}

% =========================================
% Ahora el preámbulo es coherente y organizado.
% =========================================

% Carga un archivo externo con TODOS los \usepackage{},
% definiciones de colores, comandos, etc. centralizados.

% ======================================================
% 3. Inicio del documento
% ======================================================
\begin{document}

    % ------------------------------------------------------
    % 3.1 Configuración inicial (título, autor, fecha…)
    % ------------------------------------------------------
    \newcommand{\fecha}{\DTMDate{2024-12-09}}
\newcommand{\titulo}{}
\newcommand{\autor}{
	Sebastián Zúñiga
}
%Renuevo comandos
\renewcommand{\contentsname}{Contenido}
\renewcommand{\tablename}{\bfseries Tabla}
\renewcommand{\figurename}{\bfseries Figura}
\renewcommand{\thefootnote}{\textcolor{grayblack}{\arabic{footnote}}}

    % En este archivo define \title{…}, \author{…}, \date{…},
    % y ajustes de tema (colores, footline, headline, etc.)

    % ======================================================
    % 4. Portada “canónica”
    % ======================================================
    % =========================================
% 1. Información de la presentación
% =========================================
%   Coloca estas líneas EN EL PREÁMBULO, justo después de cargar
%   los paquetes y antes de \begin{document}

% Título de la presentación:
% - Opción corta para el encabezado (aparece en la barra de navegación)
% - Opción larga para la diapositiva de título
\title[NCh 440/1 – 2014]{Normativa Legal Vigente en el Transporte Vertical}

% Autor(es):
% - Opción corta para el pie de página o navegación
% - Nombre completo en la diapositiva de título
\author[S.\ Zúñiga]{Sebastián Alfredo Zúñiga Alfaro}

% Institución o entidad:
% - Texto que aparece debajo del autor en la portada
\institute[]{Capacitación}

% Fecha:
% - Opción corta (por ejemplo, día/mes/año) para la barra de navegación
% - Texto completo en la portada
\date[23/4/25]{23 de abril de 2025}

    % Diapositiva de portada estándar generada con \maketitle o \titlepage

    % ======================================================
    % 5. Portada “a medida” sin cabecera ni pie
    % ======================================================
    {
    % Dentro de un grupo local para no alterar el resto
        \setbeamertemplate{headline}{}   % Suprime la cabecera
        \setbeamertemplate{footline}{}   % Suprime el pie de página
        \begin{frame}[plain]             % “plain” = sin decoraciones
            \vspace*{-1cm}                 % Eleva el contenido
            \begin{columns}
                \column{0.45\textwidth}       % Espacio reservado para logo
                \vspace*{-6.5cm}              % Ajuste vertical del logo
                %\includegraphics[width=4cm]{Imágenes/FM}
                % Uncomment y ajusta la ruta del logo
            \end{columns}
            \titlepage                     % Genera automáticamente título, autor, fecha
        \end{frame}
    }

    % ======================================================
    % 6. Índice automático al comenzar cada sección
    % ======================================================
    \AtBeginSection[]{
        \begin{frame}{Contenido}
            \begin{multicols}{2}          % Divide el índice en 2 columnas
                \tableofcontents[currentsection]
            \end{multicols}
        \end{frame}
    }

    % ======================================================
    % 7. Contenido principal: secciones
    % ======================================================
    \begin{frame}[t]{\textbf{Uso del presente material}}
    % ---------------------------------------
    % Opciones y título de la diapositiva
    % ---------------------------------------
    % [t] : alinea el contenido arriba
    % {Uso del presente material} : título de la diapositiva

    % ---------------------------------------
    % Columnas para maquetación
    % ---------------------------------------
    \begin{columns}[onlytextwidth]  % ajusta las columnas al ancho de texto
        \column{1\textwidth}         % esta columna ocupa el 100 % del ancho
        \textbf{Para objetivos de capacitación únicamente:}\\
        \vspace{0.5cm}               % espacio vertical entre título y párrafo
        \centering
        % -------------------------------------
        % Texto justificado (requiere ragged2e)
        % -------------------------------------
        \justifying
        La información en este paquete es sólo para fines de capacitación.
        Cualquier información técnica relativa a un producto es relevante sólo
        a la capacitación, y no debe ser utilizado por el destinatario o
        transmitido a otra persona para su uso en el diseño, fabricación,
        pruebas, instalación, inspección, mantenimiento, modernización o
        reparación del producto al que se refiere.
    \end{columns}
\end{frame}



    % - Aquí cargas cada bloque de tu presentación organizado por archivos.

    % (Opcional) Slide final con portada para cierre
    \begin{frame}
        \titlepage
    \end{frame}

\end{document}

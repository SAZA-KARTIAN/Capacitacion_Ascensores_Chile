\section{Aspectos de seguridad}
\begin{frame}{Aspectos de seguridad}
	\begin{itemize}
		\item ¿Cuál es la altura máxima, de los botones de comando, en la botonera de cabina?
		\item ¿Qué distancia debe existir entre elementos móviles a un ascensor adyacente?
		\item ¿Cuántos lux deben existir en el recorrido?
		\item ¿A qué distancia debe quedar el botón de la parada de emergencia, en un equipo sin sala de máquinas?
		\item ¿Desde qué dimensión de cabina se debe instalar un espejo?
	\end{itemize}
\end{frame}

%------------------------------------------------
\section{Preliminar}
\begin{frame}{Preliminar}
	% Slide original sin contenido adicional
\end{frame}

%------------------------------------------------
\section{Contenido \& Objetivo}
\begin{frame}{Contenido \& Objetivo}
	% En la diapositiva original aparecían los títulos "Contenido" y "Objetivo"
	% Sin lista explícita, se conservan como título único.
\end{frame}

%------------------------------------------------
\section{Bibliografía}
\begin{frame}{Bibliografía}
	% Referencias originales en la diapositiva
\end{frame}

%------------------------------------------------
\section{Registro nacional de profesionales}
\begin{frame}{Registro nacional de instaladores, mantenedores y certificadores}
	\begin{itemize}
		\item Instaladores I, II y III categoría
		\item Mantenedores única categoría
		\item Certificadores I y II categoría
	\end{itemize}
\end{frame}

%------------------------------------------------
\section{Normas técnicas}
\begin{frame}{Normas técnicas}
	\begin{itemize}
		\item Norma técnica NTM/NCh
		\item Norma técnica MINVU
		\item Norma Chilena INN
		\item Elevador debe cumplir norma técnica vigente para certificarse
	\end{itemize}
\end{frame}

%------------------------------------------------
\section{LGUC}
\begin{frame}{Requisitos de la Ley General de Urbanismo y Construcción}
	L.G.U.C. 4.1.7 numeral 3 y 4.1.11\,f
\end{frame}

%------------------------------------------------
\section{NCh 440/1:2014}
\subsection{Punto 5: Caja de elevadores}
\begin{frame}{Caja de elevadores NCh 440/1:2014. Punto 5}
	% Primer pase de diapositiva
\end{frame}

\begin{frame}{Caja de elevadores NCh 440/1:2014. Punto 5}
	% Diapositiva duplicada en original
\end{frame}

\subsection{Ejercicio intermedio}
\begin{frame}{Analice cuanto avanza su conocimiento!}
	Comente los aspectos normativos asociados a las imágenes.
	
	% Marcadores para imágenes originales
	\begin{center}
		% Aquí se insertan las imágenes de apoyo
		% \includegraphics[width=0.45\textwidth]{imagen1.ext}
		% \includegraphics[width=0.45\textwidth]{imagen2.ext}
	\end{center}
\end{frame}

\subsection{Punto 6: Sala de máquinas y cuadro de control}
\begin{frame}{Sala de máquinas y cuadro de control NCh 440/1:2014. Punto 6}
\end{frame}

\begin{frame}{Sala de máquinas y cuadro de control NCh 440/1:2014. Punto 6}
	% Diapositiva duplicada en original
\end{frame}

\subsection{Punto 7: Puertas de acceso en pisos}
\begin{frame}{Puertas de acceso en pisos NCh 440/1:2014. Punto 7}
\end{frame}

\subsection{Punto 8: Cabina y contrapeso}
\begin{frame}{Cabina y contrapeso NCh 440/1:2014. Punto 8}
\end{frame}

\begin{frame}{Analice cuanto avanza su conocimiento!}
	¿Cuál es el tamaño de la ventilación del cuarto de máquinas?
	
	\bigskip
	\textbf{0,1\,m² \quad (20×50\,cm²)}
\end{frame}

\begin{frame}{Cabina y contrapeso NCh 440/1:2014. Punto 8}
	% Diapositiva duplicada en original
\end{frame}

\subsection{Punto 9: Suspensión y seguridad}
\begin{frame}{Suspensión, compensación, paracaídas, limitador de velocidad.\\NCh 440/1:2014. Punto 9}
\end{frame}

\begin{frame}{Suspensión, compensación, paracaídas, limitador de velocidad.\\NCh 440/1:2014. Punto 9}
	% Diapositiva duplicada en original
\end{frame}

\subsection{Punto 10: Guías y amortiguadores}
\begin{frame}{Guías, amortiguadores y dispositivo de final de recorrido.\\NCh 440/1:2014. Punto 10}
\end{frame}

\subsection{Punto 11: Holguras}
\begin{frame}{Holguras entre cabina y paredes de la caja de elevadores,\\y entre la cabina y el contrapeso. NCh 440/1:2014. Punto 11}
\end{frame}

\subsection{Punto 12: Máquina}
\begin{frame}{Máquina NCh 440/1:2014. Punto 12}
\end{frame}

\subsection{Punto 13: Instalación y aspectos eléctricos}
\begin{frame}{Instalación y aspectos eléctricos. NCh 440/1:2014. Punto 13}
\end{frame}

\begin{frame}{Instalación y aspectos eléctricos. NCh 440/1:2014. Punto 13}
	% Diapositiva duplicada en original
\end{frame}

\subsection{Punto 14: Protección eléctrica}
\begin{frame}{Protección contra fallas eléctricas, controles, prioridades.\\NCh 440/1:2014. Punto 14}
\end{frame}

\subsection{Punto 15: Rótulo de instrucciones}
\begin{frame}{Rótulo de instrucciones de operaciones.\\NCh 440/1:2014. Punto 15}
\end{frame}

\begin{frame}{Rótulo de instrucciones de operaciones.\\NCh 440/1:2014. Punto 15}
	% Diapositiva duplicada en original
\end{frame}

\subsection{Punto 16: Inspecciones y mantenimiento}
\begin{frame}{Inspecciones, ensayos, registros, mantenimiento.\\NCh 440/1:2014. Punto 16}
\end{frame}

%------------------------------------------------
\section{Otras normas}
\begin{frame}{Requisitos mínimos de diseño, instalación y operación para ascensores\\electromecánicos frente a sismos. NCh3362:2014}
\end{frame}

\begin{frame}{OGUC (D 47)}
	“…ser comprensibles, utilizables y practicables por todas las personas,\\
	en condiciones de seguridad y comodidad, de la forma más autónoma\\
	y natural posible.”
\end{frame}

\begin{frame}{Inclusión Ley 20.422}
	Si las edificaciones y obras señaladas en este inciso contaren con ascensores, estos deberán tener capacidad suficiente para transportar a las personas con discapacidad de conformidad a la normativa vigente.
\end{frame}

\begin{frame}{Regulación de ascensores regenerativos}
	Los ascensores electromecánicos verticales que requieran instalarse en los edificios señalados en los números 1) y 2) de la letra a) de este numeral, corresponderán a ascensores del tipo regenerativo, entendiéndose por éstos a aquellos que durante el movimiento ascendente o descendente de la cabina o el contrapeso generan energía eléctrica. Igualmente, se conformarán como ascensores del tipo regenerativo aquellos ascensores electromecánicos instalados en estos mismos edificios, que en su alteración o transformación, consideren el cambio de la velocidad nominal, la carga nominal y/o la masa de la cabina; o consideren el cambio o sustitución de la máquina o el sistema de control.
\end{frame}

\begin{frame}{Edificios afectados}
	Edificios de vivienda de más de 6 niveles, sin tener recinto de uso exclusivo.\\
	Destinados a otros usos de 5 o más niveles.
\end{frame}

\begin{frame}{Regenerativo}
	% Título único, sin contenido adicional
\end{frame}